\chapter{Literature Review}


\textbf{Add the summaries of research work you referred for your work. Either discuss the research in chronological order or by topic. Do cite all the research articles at the end of report in Bibliography section. For example the below paragraphs discuss the research in a flow of the topic} 


It has been very crucial task for researchers to identify the current and historical trends. Further, extraction from this information to focus on innovative methods and concepts also requires critical thinking. Keeping in view the volume of unstructured data, it is a tedious task to extract the desired information. But with technological advances, various methods have been developed for extracting the relevant information from voluminous data. The data from various bibliographic sources  viz. research articles, book chapters, patents, and technical reports, can be summarized by identification of topics ~\cite{Mei2005,Hurtado2016}.

%	Natural language processing provides a powerful algorithm that extracts unobserved trends from a large collection of documents. 
Topic modeling is an automated process which is based on algorithmic-based analysis~\cite{Canini2009,Saini2013}. From a corpus, this method is used to identify the patterns. Further, semantic meaning is also added to the corpus's vocabulary. Topic modeling can be applied by two methods viz. topic analysis and clustering but topic analysis is considered better choice for research trends' identification~\cite{Evangelopoulos2012}. The main difference between the two methods is the assignment of a document to the topic or cluster. A document can be assigned to multiple topics in topic analysis, whereas it joins exactly one cluster in clustering. 


\section{Comparison}
This section should have table where you need to compare the results from different prominent research of area of interest. 
\section{Objectives of Project (Must be clearly, precisely defined and Implementation must be done.)}
This section should elaborate the objectives based on the gaps from the previous researchers.
